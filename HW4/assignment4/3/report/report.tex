\documentclass[a4paper,12pt]{article}
\usepackage{xcolor}
\usepackage{amsmath,amsfonts,amssymb,amsthm}
\usepackage{hyperref}
\usepackage{geometry}
\usepackage{fancyhdr}
\usepackage{graphicx}
\usepackage{titlesec}
\usepackage{tikz}
\usepackage{booktabs}
\usepackage{array}
\usetikzlibrary{shadows}
\usepackage{tcolorbox}
\usepackage{float}
\usepackage{lipsum}
\usepackage{mdframed}
\usepackage{pagecolor}
\usepackage{mathpazo}   % Palatino font (serif)
\usepackage{microtype}  % Better typography

% Page background color
\pagecolor{gray!10!white}

% Geometry settings
\geometry{margin=0.5in}
\pagestyle{fancy}
\fancyhf{}

% Fancy header and footer
\fancyhead[C]{\textbf{\color{blue!80}CS663 Assignment-4}}
% \fancyhead[R]{\color{blue!80}Saksham Rathi}
\fancyfoot[C]{\thepage}

% Custom Section Color and Format with Sans-serif font
\titleformat{\section}
{\sffamily\color{purple!90!black}\normalfont\Large\bfseries}
{\thesection}{1em}{}

% Custom subsection format
\titleformat{\subsection}
{\sffamily\color{cyan!80!black}\normalfont\large\bfseries}
{\thesubsection}{1em}{}

% Stylish Title with TikZ (Enhanced with gradient)
\newcommand{\cooltitle}[1]{%
  \begin{tikzpicture}
    \node[fill=blue!20,rounded corners=10pt,inner sep=12pt, drop shadow, top color=blue!50, bottom color=blue!30] (box)
    {\Huge \bfseries \color{black} #1};
  \end{tikzpicture}
}
\usepackage{float} % Add this package

\newenvironment{solution}[2][]{%
    \begin{mdframed}[linecolor=blue!70!black, linewidth=2pt, roundcorner=10pt, backgroundcolor=yellow!10!white, skipabove=12pt, skipbelow=12pt]%
        \textbf{\large #2}
        \par\noindent\rule{\textwidth}{0.4pt}
}{
    \end{mdframed}
}

% Document title
\title{\cooltitle{CS663 Assignment-4}}
\author{{\bf Saksham Rathi, Kavya Gupta, Shravan Srinivasa Raghavan} \\
\small Department of Computer Science, \\
Indian Institute of Technology Bombay \\}
\date{}

\newtheorem{Def}{Definition}
\newtheorem{lemma}{Lemma}
\newtheorem{theorem}{Theorem}

\begin{document}
\maketitle

\section*{Question 3}

\begin{solution}{Solution}
    Let $A$ be a real $m \times n$ matrix. $A$ can always be expressed as $A = U \Sigma V^{T}$ where 
    $U \in \mathbb{R}^{m \times m}$, $V \in \mathbb{R}^{n \times n}$ and $\Sigma \in \mathbb{R}^{m \times n}$, 
    $U,V$ being orthogonal matrices and $\Sigma$ being a diagonal matrix with non negative values (called singular values)
    on the diagonal. Let $\Sigma = diag(\sigma_{1},\sigma_{2}, \dots,\sigma_{\min(m,n)})$. We will show that the squares 
    of the non-zero singular values of $A$ are the eigenvalues of either $AA^{T}$ or $A^{T}A$. This is equivalent to 
    showing that the the non-zero singular values of $A$ are the squareroots of either $AA^{T}$ or $A^{T}A$ because,
    the singular values are non-negative by definition. 
    \section*{Part a}
      We will show that the non zero singular values of $A$ are equal to the square roots of the eigen values of either 
      $AA^{T}$ or $A^{T}A$. We define $\Sigma_{m}^{2} = \Sigma \Sigma^{T}$ and $\Sigma^{2}_{n} = \Sigma^{T} \Sigma$.

      \begin{align*}
        AA^{T} &= (U \Sigma V^{T}) \cdot {(U \Sigma V^{T})}^{T} \\
               &= U \Sigma V^{T} \cdot (V \Sigma^{T} U^{T}) \\
               &= U \Sigma V^{T} V \Sigma^{T} U^{T} \\
               &= U \Sigma (V^{T}V) \Sigma^{T} U^{T} \\ 
               &= U \Sigma I_{n} \Sigma ^{T} U^{T} \\
               &= U \Sigma \Sigma^{T} U^{T} \\
               &= U \Sigma^{2}_{m} U^{T} \\
      \end{align*}

      Similarly, $A^{T}A = V \Sigma^{T} \Sigma V^{T} = V \Sigma_{n}^{2} V^{T}$.

      Clearly, $\Sigma^{2}_{m} \in R^{m \times m}$ and equals $diag(\sigma_{1}^{2},\sigma_{2}^{2},\dots , \sigma_{m}^{2}) = D$.
      Now $m = \min(m,n)$ or $n = \min(m,n)$.  

      Let $m = \min(m,n)$. In this case, $\Sigma_{m}^{2}$ is a diagonal matrix ($D$) whose entries are squares of the singular 
      values of $A$. Since $AA^{T}$ is a real symmetric matrix, by \textbf{spectral theorem}  it has an orthogonal decomposition given by 
      $AA^{T} = \mathcal{U} \mathcal{D} \mathcal{U^{T}}$ where $\mathcal{D}$ is a diagonal matrix whose entries are the eigenvalues
      of $AA^{T}$ and $\mathcal{U}$ is an orthogonal matrix. Therefore the non zero entries of $D$ are the eigenvalues of $AA^{T}$.
      Since $D = \Sigma_{m}^{2}$, the non zero entries of $D$ are the squares of the non zero singular values of $A$ and the squares of the any 
      non-zero singular value of $A$ is an eigenvalue of $AA^{T}$, we are done.

      Let $n = \min(m,n)$. In this case, we deal with the diagonal matrix $D = \Sigma_{n}^{2}$ whose entries are the squares of 
      all the singular values of $A$ (this is because $n = \min(m,n)$). The proof is very similar to the case above. Since
      $A^{T}A$ is a real symmetric matrix, it has an orthogonal decomposition $\mathcal{V} \mathcal{D} \mathcal{V}^{T}$
      and therefore we have the non-zero entries of $D$ to be the eigenvalues of $A^{T}A$ and following a similar argument we 
      arrive at the conclusion that the square of any non-zero singular value of $A$ is an eigen value of $A^{T}A$ and that
      any eigenvalue of $A^{T}A$ is the square of some non-zero singular value of $A$. This completes the proof.
      
      \section*{Part b}

      The Frobenius norm of a matrix $A \in \mathbb{R}^{m \times n}$ is defined as 
      \[\left \lvert \lvert A \right \rvert \rvert_{F} = \sqrt{\left(\sum\limits_{i = 1}^{m} \sum\limits_{j = 1}^{n} A_{ij}^{2}\right)} \]

      For any matrix $A \in \mathbb{R}^{m \times n}$ we have 
      \[ \sum\limits_{i = 1}^{m}\sum\limits_{j = 1}^{n} A_{ij}^{2} = \text{Tr}(AA^{T}) = \text{Tr}(A^{T}A)\] where $\text{Tr}(M) = \sum\limits_{i = 1}^{n} M_{ii}$
      for any $n \times n$ square matrix $M$. The trace of a matrix also has the following property,
      \[\text{Tr}(AB) = \text{Tr}(BA)\] whenever $AB$ and $BA$ are both square matrices for two matrices $A$ and $B$.

      WLOG we take $m = \min(n,m)$ (in the other case we simply deal with $A^{T}A$ and $\Sigma_{n}^{2}$)
      \begin{align*}
          \lvert\lvert A \rvert\rvert_{F}^{2} &= \sum\limits_{i = 1}^{m}\sum\limits_{j = 1}^{n} A_{ij}^{2} \\
                                              &= \text{Tr}(AA^{T}) \\
                                              &= \text{Tr}((U\Sigma V^{T} V \Sigma^{T} U^{T})) \\
                                              &= \text{Tr} (U \Sigma_{m}^{2} U^{T}) \\
                                              &= \text{Tr} ((U \Sigma_{m}^{2}) (U^{T})) \\
                                              &= \text{Tr} ((U^{T})(U \Sigma_{m}^{2})) \\
                                              &= \text{Tr} (U^{T} U \Sigma_{m}^{2}) \\
                                              &= \text{Tr} (I_{m} \Sigma_{m}^{2}) \\
                                              &= \text{Tr} (\Sigma_{m}^{2}) \\
                                              &= \sum\limits_{i = 1}^{k} \sigma_{i}^{2} \,\text{ (where k is the number of non-zero singular values of A)} \\
          \Rightarrow \lvert\lvert A \rvert \rvert_{F}^{2} &= \sum\limits_{i = 1}^{k} \sigma_{i}^{2} 
      \end{align*}

      Thus we have the square of the Frobenius norm of a matrix to equal the sum of squares of the singular values.

      \section*{Part c}

      \section*{Part d}

      Given:
      \begin{align*}
        A \in \mathbb{R}^{m \times n} \, &, m \leqslant n \\
        P &= A^{T}A \\
        Q &= AA^{T}
      \end{align*}

      A few results and definitions we will use:
      \begin{Def}\label{psd}
         A matrix $M \in \mathbb{R}^{n \times n}$ is said to be positive semi-definite if $\forall \vec{x} \in R^{n}$,
         \[ \vec{v}^{T} M \vec{v} \geqslant 0\]
      \end{Def}
      \begin{theorem}[Spectral Theorem]\label{spectral}
        Any real symmetric matrix $A \in \mathbb{R}^{n \times n}$ is orthogonally diagonalizable. That is there exists real numbers
        $\lambda_{1},\lambda_{2}, \dots,\lambda_{n}$, $U \in \mathbb{R}^{n \times n}$ such that $A = UDU^{T}$ where 
        $D = \text{diag}(\lambda_{1},\lambda_{2},\dots,\lambda_{n})$,
        $U = \left[ \vec{u}_{1} \vec{u}_{2} \cdots \vec{u}_{n}\right]$ with 
        $A \vec{u}_{i} = \lambda_{i} \vec{u}_{i} \, \forall i \in [1,n]$ and $UU^{T} = U^{T}U = I_{n}$
      \end{theorem}
        \begin{lemma}\label{l1}
        $P$ and $Q$ are positive semi-definite (defintion \@\ref{psd}) and their eigenvalues are non-negative.
          \begin{proof}
            We will first show that $P$ is positive semi-definite and then we will show that the eigen values of $P$ 
            are all non-negative. The proof for $Q$ is similar.     
            
            Let $\vec{x} \in \mathbb{R}^{n}$. 
            \begin{align}
            \vec{x}^{T} P \vec{x} &= \vec{x}^{T}(A^{T}A) \vec{x} \nonumber \\ 
                                  &= \vec{x}^{T} A^{T} A \vec{x} \nonumber \\
                                  &= (\vec{x}^{T} A^{T}) (A \vec{x}) \nonumber \\
                                  &= {(A \vec{x})}^{T} (A\vec{x}) \nonumber \\
                                  &= \vec{y}^{T} \vec{y} \, \text{ where $\vec{y} \in \mathbb{R}^{n} \, \vec{y} = A \vec{x}$} \nonumber \\
                                  &= \lvert \lvert \vec{y} \rvert \rvert^{2} \geqslant 0 \nonumber \\
            \Rightarrow \label{r1}\vec{x}^{T} P \vec{x} &\geqslant 0
            \end{align}
            
            Let $\vec{u} \in R^{n}/\{\vec{0}\} $ be an eigenvector of $P$ with corresponding eigenvalue $\lambda \in \mathbb{C}$.
            Consider the expression $\vec{u}^{T} P \vec{u}$. From equation\@ref{1} we know that $\vec{u}^{T} P \vec{u} \geqslant 0$.
            Therefore we have,

            \begin{align}
               \vec{u}^{T} P \vec{u} &= \vec{u}^{T}(P \vec{u}) \geqslant 0 \nonumber  \\
                                     = \vec{u}^{T}(\lambda \vec{u}) &\geqslant 0 \nonumber \\
                                     = \lambda \vec{u}^{T} \vec{u} &\geqslant 0\nonumber \\
                                     = \lambda \lvert\lvert \vec{u} \rvert\rvert^{2} &\geqslant 0 \nonumber \\
                \Rightarrow \lambda \lvert\lvert \vec{u} \rvert\rvert^{2} &\geqslant 0 \nonumber \\
                \Rightarrow \label{r2}\lambda &\geqslant 0
            \end{align}

            Therefore we have eigenvalues of $P$ to be non-negative. The proofs hold for $Q$ as well, just replace $\vec{x}$ in
            equation \@\ref{r1} by $\vec{y} \in \mathbb{R}^{m}$ and $\vec{u}$ in equation \@\ref{r2} by
            $\vec{v} \in \mathbb{R}^{m}$.
          \end{proof}
        \end{lemma}

        \begin{lemma}\label{l2}The following are true regarding the eigenvectors of $P$ and $Q$:
          \begin{enumerate}
            \item If $\vec{u} \in \mathbb{R}^{n}$ is an eigenvector of $P$ with eigenvalue $\lambda$ then $A\vec{u}$ is an eigenvector of 
            $Q$ with eigenvalue $\lambda$.
            \item If $\vec{v} \in \mathbb{R}^{m}$ is an eigenvector of $Q$ with eigenvalue $\mu$, then $A^{T} \vec{v}$ is an eigenvector of $P$ with eigenvalue $\mu$.
        \end{enumerate}
          \begin{proof}
            Since $\vec{u} \in \mathbb{R}^{n}$ is an eigenvector of $P$ with eigenvalue $\lambda$ we have,
            \begin{align}
              P\vec{u} &= \lambda \vec{u} \nonumber \\
              \Rightarrow P\vec{u} &= (A^{T}A) \vec{u} = \lambda \vec{u} \nonumber \\
              \Rightarrow (A^{T}) (A \vec{u}) &= \lambda \vec{u} \nonumber \\
              \Rightarrow A \left( A^{T} (A\vec{u})\right) &= A \left( \lambda \vec{u}\right) \nonumber \, \text{ (pre-multiplying both LHS and RHS by a non-null matrix $A$)} \nonumber \\
              \Rightarrow  (A A^{T}) (A\vec{u}) &= \lambda (A \vec{u}) \nonumber \\
              \Rightarrow (Q) (A\vec{u}) &= \lambda (A\vec{u}) \nonumber \\
              \Rightarrow  \label{r3}Q (A\vec{u}) &= \lambda (A\vec{u})
            \end{align}
            From equation \@\ref{r3} we have that for every $\vec{u}$ that is eigenvector of $P$, the vector $A\vec{u}$
            is an eigenvector of $Q$ with the same eigenvalue. Now we prove the second part of the lemma whose proof is very
            similar to what we saw above.

            Since $\vec{v} \in \mathbb{R}^{m}$ is an eigenvector of $Q$ with eigenvalue $\mu$ we have,
            \begin{align}
              Q\vec{v} &= \mu \vec{v} \nonumber \\
              \Rightarrow Q\vec{v} &= (AA^{T}) \vec{v} = \mu \vec{v} \nonumber \\
              \Rightarrow (A) (A^{T} \vec{v}) &= \mu \vec{v} \nonumber \\
              \Rightarrow A^{T} \left( A (A^{T}\vec{v})\right) &= A^{T} \left( \mu \vec{v}\right) \nonumber \, \text{ (pre-multiplying both LHS and RHS by a non-null matrix $A^{T}$)} \nonumber \\
              \Rightarrow  (A^{T} A) (A^{T}\vec{v}) &= \mu (A^{T} \vec{v}) \nonumber \\
              \Rightarrow (P) (A^{T}\vec{v}) &= \mu (A^{T}\vec{v}) \nonumber \\
              \Rightarrow  \label{r4}P (A^{T}\vec{v}) &= \mu (A^{T}\vec{v})
            \end{align}

            Equations \@\ref{r3} and \@\ref{r4} complete the proof.
          \end{proof}
        \end{lemma}

        \begin{lemma}\label{l3}
          Let $\vec{v}_{i} \in \mathbb{R}^{m}$ be an eigenvector of $Q$. Define 
          $u_{i} \stackrel{\Delta}{=} \frac{A^{T} \vec{v}_{i}}{\lvert\lvert A^{T} \vec{v}_{i} \rvert\rvert_{2}}$. There exists 
          $\gamma_{i} \geqslant 0$ such that $A\vec{u_{i}} = \gamma_{i} \vec{v}_{i}$
          \begin{proof} % We will use the results of lemmas \@\ref{l1} and \@\ref{l2} to prove the statement of this lemma.
            % \begin{enumerate}
            %   \item From lemma \@\ref{l2} we know that if $\vec{v}_{i}$ is eigenvector of $Q$ with eigenvalue $\mu_{i}$, then
            %   $A^{T}\vec{v}_{i}$ is eigenvector of $P$ with eigenvalue $\mu_{i}$.
            %   \item From lemma \@\ref{l1} we know that eigenvalues of $P$ and $Q$ are non-negative, therefore we have 
            %   $\mu_{i} \geqslant 0$.
            %   \item For any matrix $M$ with an eigenvector $\vec{x}$ and corresponding eigenvalue $\nu$, we always have $c\vec{x}$ 
            %   to also be an eigenvector with eigenvalue $\nu$ for any real $c \neq 0$.
            % \end{enumerate}
            % Combining these three results we have 
            % $\vec{u}_{i} \stackrel{\Delta}{=} \frac{A^{T} \vec{v}_{i}}{\lvert\lvert A^{T} \vec{v}_{i} \rvert\rvert_{2}}$
            % which is a unit vector in $\mathbb{R}^{n}$ to be an eigenvector of $P$ with eigenvalue $\mu_{i}$.
            Let $\vec{v}_{i} \in \mathbb{R}^{m}$ be an eigenvector of $Q$ with an eigenvalue $\mu$. From lemma \@\ref{l1} we know that eigenvalues of $P$ and $Q$ are non-negative, therefore we have 
            $\mu_{i} \geqslant 0$.
            Consider the expression $A\vec{u}_{i}$,
            \begin{align}
               A\vec{u}_{i} &= A \left(\frac{A^{T} \vec{v}_{i}}{\lvert\lvert A^{T} \vec{v}_{i} \rvert\rvert_{2}}\right) \nonumber \\
                            &= \frac{ A (A^{T} \vec{v}_{i})}{\lvert\lvert A^{T} \vec{v}_{i} \rvert\rvert_{2}} \nonumber \\
                            &= \frac{ (A A^{T}) \vec{v}_{i}}{\lvert\lvert A^{T} \vec{v}_{i} \rvert\rvert_{2}} \nonumber \\
                            &= \frac{Q\vec{v}_{i}}{\lvert\lvert A^{T} \vec{v}_{i} \rvert\rvert_{2}} \nonumber \\
                            &= \frac{\mu_{i}\vec{v}_{i}}{\lvert\lvert A^{T} \vec{v}_{i} \rvert\rvert_{2}} \nonumber \\
                            &= \left(\frac{\mu_{i}}{\lvert\lvert A^{T} \vec{v}_{i} \rvert\rvert_{2}}\right) \vec{v}_{i} \nonumber \\
                            &= \gamma_{i} \vec{v}_{i}
            \end{align}

            That is, there exists 
            $\gamma_{i} = \left(\frac{\mu_{i}}{\lvert\lvert A^{T} \vec{v}_{i} \rvert\rvert_{2}}\right) \geqslant 0$ 
            such that $A\vec{u}_{i} = \gamma_{i} \vec{v}_{i}$
          \end{proof}
        \end{lemma}

        \begin{theorem}[Singular Value Decomposition]
            Let $\vec{v}_{i}$,$\vec{u_{i}}$ and $\gamma_{i}$ be defined as in lemma \@\ref{l3}. 
            Let $U = \left[\vec{v}_{1} | \vec{v}_{2} | \cdots | \vec{v}_{m}\right]$,
            $V = \left[\vec{u}_{1} | \vec{u}_{2} | \cdots | \vec{u}_{m}\right]$ and 
            $\Gamma =\text{diag}(\gamma_{1},\gamma_{2},\dots,\gamma_{m})$ where $U$ and $\Gamma$ are both $m \times m$
            matrices and $V$ is an $n\times m$ matrix. Then \[ A = U \Gamma V^{T} \]
            \begin{proof}
              Two matrices $A$ and $B$ are equal only if they the same dimensions and the values at the corresponding indices
              for every pair of $(i,j)$ are equal ie, $\forall i \in [1,m]$, $\forall j \in [1,n]$, if $A_{ij} = B_{ij}$
              then $A = B$ and vice versa.

              As $P$ and $Q$ are real symmetric matrices we can obtain an orthonormal basis for $\mathbb{R}^{n}$ and 
              $\mathbb{R}^{m}$ respectively that are the eigenvectors of $P$ and $Q$ respectively (theorem \@\ref{spectral}).

              Since the columns of $U$ and $V$ are the orthonormal eigenvectors of $Q$ and $P$ respectively, we have 
              $UU^{T} = U^{T}U = I_{m}$ and $V^{T}V = I_{m}$.
              
              Consider the expression $AV$.
              \begin{align}
                AV &= A \left[ \vec{u}_{1} | \vec{u}_{2} | \cdots | \vec{u}_{m}\right] \nonumber \\
                \Rightarrow AV &= \left[ A\vec{u}_{1} | A\vec{u}_{2} | \cdots | A \vec{u}_{m} \right] \nonumber \\
                               &= \left[ \gamma_{1}\vec{v}_{1} | \gamma_{2}\vec{v}_{2} | \cdots | \gamma_{m} \vec{v}_{m} \right] \text{ from lemma \@\ref{l3}} \nonumber \\
                               &= \left[\vec{v}_{1} | \vec{v}_{2} | \cdots | \vec{v}_{m}\right] \Gamma \nonumber \\
                               &= U \Gamma \nonumber \\
                \Rightarrow AV &= U \Gamma \nonumber \\
                \Rightarrow A V V^{T} &= \text{ if this is somehow equal to $I_{n}$ then we are done.}
              \end{align}
            \end{proof}
        \end{theorem}
  \end{solution}
\end{document}
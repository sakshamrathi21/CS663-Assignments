\documentclass[a4paper,12pt]{article}
\usepackage{xcolor}
\usepackage{amsmath,amsfonts,amssymb}
\usepackage{geometry}
\usepackage{fancyhdr}
\usepackage{graphicx}
\usepackage{titlesec}
\usepackage{tikz}
\usepackage{booktabs}
\usepackage{array}
\usetikzlibrary{shadows}
\usepackage{tcolorbox}
\usepackage{float}
\usepackage{lipsum}
\usepackage{mdframed}
\usepackage{pagecolor}
\usepackage{mathpazo}   % Palatino font (serif)
\usepackage{microtype}  % Better typography

% Page background color
\pagecolor{gray!10!white}

% Geometry settings
\geometry{margin=0.5in}
\pagestyle{fancy}
\fancyhf{}

% Fancy header and footer
\fancyhead[C]{\textbf{\color{blue!80}CS663 Assignment-4}}
% \fancyhead[R]{\color{blue!80}Saksham Rathi}
\fancyfoot[C]{\thepage}

% Custom Section Color and Format with Sans-serif font
\titleformat{\section}
{\sffamily\color{purple!90!black}\normalfont\Large\bfseries}
{\thesection}{1em}{}

% Custom subsection format
\titleformat{\subsection}
{\sffamily\color{cyan!80!black}\normalfont\large\bfseries}
{\thesubsection}{1em}{}

% Stylish Title with TikZ (Enhanced with gradient)
\newcommand{\cooltitle}[1]{%
  \begin{tikzpicture}
    \node[fill=blue!20,rounded corners=10pt,inner sep=12pt, drop shadow, top color=blue!50, bottom color=blue!30] (box)
    {\Huge \bfseries \color{black} #1};
  \end{tikzpicture}
}
\usepackage{float} % Add this package

\newenvironment{solution}[2][]{%
    \begin{mdframed}[linecolor=blue!70!black, linewidth=2pt, roundcorner=10pt, backgroundcolor=yellow!10!white, skipabove=12pt, skipbelow=12pt]%
        \textbf{\large #2}
        \par\noindent\rule{\textwidth}{0.4pt}
}{
    \end{mdframed}
}

% Document title
\title{\cooltitle{CS663 Assignment-4}}
\author{{\bf Saksham Rathi, Kavya Gupta, Shravan Srinivasa Raghavan} \\
\small Department of Computer Science, \\
Indian Institute of Technology Bombay \\}
\date{}

\begin{document}
\maketitle

\section*{Question 3}

\begin{solution}{Solution}
    Let $A$ be a real $m \times n$ matrix. $A$ can always be expressed as $A = U \Sigma V^{T}$ where 
    $U \in \mathbb{R}^{m \times m}$, $V \in \mathbb{R}^{n \times n}$ and $\Sigma \in \mathbb{R}^{m \times n}$, 
    $U,V$ being orthogonal matrices and $\Sigma$ being a diagonal matrix with non negative values (called singular values)
    on the diagonal. Let $\Sigma = diag(\sigma_{1},\sigma_{2}, \dots,\sigma_{\min(m,n)})$. We will show that the squares 
    of the non-zero singular values of $A$ are the eigenvalues of either $AA^{T}$ or $A^{T}A$. This is equivalent to 
    showing that the the non-zero singular values of $A$ are the squareroots of either $AA^{T}$ or $A^{T}A$ because,
    the singular values are non-negative by definition. 
    \section*{Part a}
      We will show that the non zero singular values of $A$ are equal to the square roots of the eigen values of either 
      $AA^{T}$ or $A^{T}A$. We define $\Sigma_{m}^{2} = \Sigma \Sigma^{T}$ and $\Sigma^{2}_{n} = \Sigma^{T} \Sigma$.

      \begin{align*}
        AA^{T} &= (U \Sigma V^{T}) \cdot {(U \Sigma V^{T})}^{T} \\
               &= U \Sigma V^{T} \cdot (V \Sigma^{T} U^{T}) \\
               &= U \Sigma V^{T} V \Sigma^{T} U^{T} \\
               &= U \Sigma (V^{T}V) \Sigma^{T} U^{T} \\ 
               &= U \Sigma I_{n} \Sigma ^{T} U^{T} \\
               &= U \Sigma \Sigma^{T} U^{T} \\
               &= U \Sigma^{2}_{m} U^{T} \\
      \end{align*}

      Similarly, $A^{T}A = V \Sigma^{T} \Sigma V^{T} = V \Sigma_{n}^{2} V^{T}$.

      Clearly, $\Sigma^{2}_{m} \in R^{m \times m}$ and equals $diag(\sigma_{1}^{2},\sigma_{2}^{2},\dots , \sigma_{m}^{2}) = D$.
      Now $m = \min(m,n)$ or $n = \min(m,n)$.  

      Let $m = \min(m,n)$. In this case, $\Sigma_{m}^{2}$ is a diagonal matrix ($D$) whose entries are squares of the singular 
      values of $A$. Since $AA^{T}$ is a real symmetric matrix, by \textbf{spectral theorem}  it has an orthogonal decomposition given by 
      $AA^{T} = \mathcal{U} \mathcal{D} \mathcal{U^{T}}$ where $\mathcal{D}$ is a diagonal matrix whose entries are the eigenvalues
      of $AA^{T}$ and $\mathcal{U}$ is an orthogonal matrix. Therefore the non zero entries of $D$ are the eigenvalues of $AA^{T}$.
      Since $D = \Sigma_{m}^{2}$, the non zero entries of $D$ are the squares of the non zero singular values of $A$ and the squares of the any 
      non-zero singular value of $A$ is an eigenvalue of $AA^{T}$, we are done.

      Let $n = \min(m,n)$. In this case, we deal with the diagonal matrix $D = \Sigma_{n}^{2}$ whose entries are the squares of 
      all the singular values of $A$ (this is because $n = \min(m,n)$). The proof is very similar to the case above. Since
      $A^{T}A$ is a real symmetric matrix, it has an orthogonal decomposition $\mathcal{V} \mathcal{D} \mathcal{V}^{T}$
      and therefore we have the non-zero entries of $D$ to be the eigenvalues of $A^{T}A$ and following a similar argument we 
      arrive at the conclusion that the square of any non-zero singular value of $A$ is an eigen value of $A^{T}A$ and that
      any eigenvalue of $A^{T}A$ is the square of some non-zero singular value of $A$. This completes the proof.
      
      \section*{Part b}

      The Frobenius norm of a matrix $A \in \mathbb{R}^{m \times n}$ is defined as 
      \[\left \lvert \lvert A \right \rvert \rvert_{F} = \sqrt{\left(\sum\limits_{i = 1}^{m} \sum\limits_{j = 1}^{n} A_{ij}^{2}\right)} \]

      For any matrix $A \in \mathbb{R}^{m \times n}$ we have 
      \[ \sum\limits_{i = 1}^{m}\sum\limits_{j = 1}^{n} A_{ij}^{2} = \text{Tr}(AA^{T}) = \text{Tr}(A^{T}A)\] where $\text{Tr}(M) = \sum\limits_{i = 1}^{n} M_{ii}$
      for any $n \times n$ square matrix $M$. The trace of a matrix also has the following property,
      \[\text{Tr}(AB) = \text{Tr}(BA)\] whenever $AB$ and $BA$ are both square matrices for two matrices $A$ and $B$.

      WLOG we take $m = \min(n,m)$ (in the other case we simply deal with $A^{T}A$ and $\Sigma_{n}^{2}$)
      \begin{align*}
          \lvert\lvert A \rvert\rvert_{F}^{2} &= \sum\limits_{i = 1}^{m}\sum\limits_{j = 1}^{n} A_{ij}^{2} \\
                                              &= \text{Tr}(AA^{T}) \\
                                              &= \text{Tr}((U\Sigma V^{T} V \Sigma^{T} U^{T})) \\
                                              &= \text{Tr} (U \Sigma_{m}^{2} U^{T}) \\
                                              &= \text{Tr} ((U \Sigma_{m}^{2}) (U^{T})) \\
                                              &= \text{Tr} ((U^{T})(U \Sigma_{m}^{2})) \\
                                              &= \text(Tr) (U^{T} U \Sigma_{m}^{2}) \\
                                              &= \text(Tr) (I_{m} \Sigma_{m}^{2}) \\
                                              &= \text(Tr) (\Sigma_{m}^{2}) \\
                                              &= \sum\limits_{i = 1}^{k} \sigma_{i}^{2} \,\text{ (where k is the number of non-zero singular values of A)} \\
          \Rightarrow \lvert\lvert A \rvert \rvert_{F}^{2} &= \sum\limits_{i = 1}^{k} \sigma_{i}^{2}
      \end{align*}

\end{solution}




\end{document}

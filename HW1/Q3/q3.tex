\documentclass[12pt]{article}

\usepackage{geometry}
\geometry{a4paper, left=1in, right=1in, top=1in, bottom=1in}
\usepackage{amsmath}
\usepackage{amsmath,amsfonts,amssymb}
\usepackage{graphicx}
\usepackage{enumitem}
\usepackage{titlesec}
\usepackage{fancyhdr}
\usepackage{hyperref}
\usepackage{floatrow}

\begin{document}
    \begin{itemize}
        \item Since we have access to the coordinates of both MATLAB's coordinate system and that of the image, we can compute the 
        transformation matrix between the two coordinate systems. Since the graph is not rotated, the equations are fairly 
        simple.
        \item The only possible relative motions are shifting of the orgin and non uniform scaling.
    \end{itemize}

    Therefore we have 
    \begin{align}
        \begin{pmatrix}
            x_{2} \\
            y_{2} \\
            1
        \end{pmatrix}
        = \mathcal{T} 
        \begin{pmatrix}
            x_{1} \\
            y_{1} \\
            1
        \end{pmatrix}
    \end{align}
    where $\mathcal{T} = \begin{pmatrix}
                            c_{1} & 0 & t_{x} \\
                            0  & c_{2} & t_{y} \\
                            0 & 0 & 1
                        \end{pmatrix}$

    The transformation represented by $\mathcal{T}$ can be solved for step by step by taking a few points from the image 
    and mapping with the coordinates in MATLAB's graph. First step would be to solve for the origins of both the systems.
    This would give us the values of $t_{x}$ and $t_{y}$. Then once that is done, we can compare the scales of both 
    the systems. The scales are basically the values $c_{1}$ and $c_{2}$ and we are done.
\end{document}
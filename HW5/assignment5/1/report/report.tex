\documentclass[a4paper,12pt]{article}
\usepackage{xcolor}
\usepackage{amsmath,amsfonts,amssymb}
\usepackage{geometry}
\usepackage{fancyhdr}
\usepackage{graphicx}
\usepackage{titlesec}
\usepackage{tikz}
\usepackage{booktabs}
\usepackage{array}
\usetikzlibrary{shadows}
\usepackage{tcolorbox}
\usepackage{float}
\usepackage{lipsum}
\usepackage{mdframed}
\usepackage{pagecolor}
\usepackage{mathpazo}   % Palatino font (serif)
\usepackage{microtype}  % Better typography

\setlength{\parindent}{0pt}

% Page background color
\pagecolor{gray!10!white}

% Geometry settings
\geometry{margin=0.5in}
\pagestyle{fancy}
\fancyhf{}

% Fancy header and footer
\fancyhead[C]{\textbf{\color{blue!80}CS663 Assignment-5}}
% \fancyhead[R]{\color{blue!80}Saksham Rathi}
\fancyfoot[C]{\thepage}

% Custom Section Color and Format with Sans-serif font
\titleformat{\section}
{\sffamily\color{purple!90!black}\normalfont\Large\bfseries}
{\thesection}{1em}{}

% Custom subsection format
\titleformat{\subsection}
{\sffamily\color{cyan!80!black}\normalfont\large\bfseries}
{\thesubsection}{1em}{}

% Stylish Title with TikZ (Enhanced with gradient)
\newcommand{\cooltitle}[1]{%
\begin{tikzpicture}
\node[fill=blue!20,rounded corners=10pt,inner sep=12pt, drop shadow, top color=blue!50, bottom color=blue!30] (box)
{\Huge \bfseries \color{black} #1};
\end{tikzpicture}
}
\usepackage{float} % Add this package

\newenvironment{solution}[2][]{%
\begin{mdframed}[linecolor=blue!70!black, linewidth=2pt, roundcorner=10pt, backgroundcolor=yellow!10!white, skipabove=12pt, skipbelow=12pt]%
	\textbf{\large #2}
	\par\noindent\rule{\textwidth}{0.4pt}
}{
\end{mdframed}
}

% Document title
\title{\cooltitle{CS663 Assignment-5}}
\author{{\bf Saksham Rathi, Kavya Gupta, Shravan Srinivasa Raghavan} \\
\small Department of Computer Science, \\
Indian Institute of Technology Bombay \\}
\date{}

\begin{document}
\maketitle

\section*{Question 1}
\begin{solution}{Part (a)}
\textbf{\Large Procedure:-}

Let the translation operation for a $(x_0, y_0)$ displacement on an image $f_1$ be defined as $f_2(x, y) = f_1(x - x_0, y - y_0)$. Given images $f_1, f_2$ have size $N \times N$, their (Discrete) Fourier Transforms are related as $F_2(\mu, \nu) = F_1(\mu, \nu) e^{-j2\pi (\mu x_0 + \nu y_0)/N}$ (due to 2D-DFT Translation property).

As per Equation (3) of Section I of the paper, cross spectrum of $f_1$ and $f_2$ is:-
\begin{equation}
	C(\mu, \nu) = \frac{F_1(\mu, \nu) \, F_2^{*}(\mu, \nu)}{|F_1(\mu, \nu)||F_2(\mu, \nu)|} = e^{j2\pi (\mu x_0 + \nu y_0)/N}
	\label{eq:cross-spectrum}
\end{equation}

\hrule
\vspace{5pt}
\textbf{Proof for Equation \ref{eq:cross-spectrum} above:-}

Given $F_2(\mu, \nu) = F_1(\mu, \nu) e^{-j2\pi (\mu x_0 + \nu y_0)/N}$, we can write $F_2^{*}(\mu, \nu) = F_1^{*}(\mu, \nu) e^{j2\pi (\mu x_0 + \nu y_0)/N}$.
\begin{align*}
	\text{Now }|F_2(\mu, \nu)|^2 &= F_2^{*}(\mu, \nu) . F_2(\mu, \nu) \\
	&= F_1(\mu, \nu) e^{-j2\pi (\mu x_0 + \nu y_0)/N} . F_1^{*}(\mu, \nu) e^{j2\pi (\mu x_0 + \nu y_0)/N} \\
	&= F_1(\mu, \nu) . F_1^{*}(\mu, \nu) = |F_1(\mu, \nu)|^2
\end{align*}
Hence $|F_2(\mu, \nu)| = |F_1(\mu, \nu)|$. Substituting these values in Equation \ref{eq:cross-spectrum}, we get:-
\[C(\mu, \nu) = \frac{F_1(\mu, \nu) \, F_1^{*}(\mu, \nu) e^{j2\pi (\mu x_0 + \nu y_0)/N}}{|F_1(\mu, \nu)|^2} = e^{j2\pi (\mu x_0 + \nu y_0)/N}\]
\hrule
\vspace{5pt}
Let's take the Inverse (Discrete) Fourier Transform of $C(\mu, \nu)$ to get the autocorrelation function $c(x, y)$:-
\begin{align*}
	c(x, y) &= \mathcal{F}^{-1}\{C(\mu, \nu)\} = \mathcal{F}^{-1}\{e^{j2\pi (\mu x_0 + \nu y_0)/N}\} \\
	&= \frac{1}{N} \sum_{\mu=0}^{N-1} \sum_{\nu=0}^{N-1} e^{j2\pi (\mu x_0 + \nu y_0)/N} e^{j2\pi (\mu x + \nu y)/N} = \frac{1}{N} \sum_{\mu=0}^{N-1} \sum_{\nu=0}^{N-1} e^{j2\pi (\mu (x_0 + x) + \nu (y_0 + y))/N} \\
	&= \delta(x_0 + x, y_0 + y)
\end{align*}
See that $c(x,y)$ attains a non-zero value only at $(-x_0, -y_0)$, which is the displacement between the two images. Hence, the autocorrelation function of the cross spectrum is a delta function at the displacement $(-x_0, -y_0)$, hence helping us find value of $(x_0, y_0)$.

Hence the steps are:-
\begin{enumerate}
	\item Obtain 2D-DFT of $f_1$ and $f_2$. $f_2$'s can be found using the Fourier Shift Theorem.
	\item Compute the cross spectrum $C(\mu, \nu)$ using Equation \ref{eq:cross-spectrum}.
	\item Determine Inverse DFT of $C(\mu, \nu)$ and find its non-zero point. If it is at $(a, b)$, then the displacement vector is $(-a, -b)$.
\end{enumerate}

\textbf{\Large Time Complexity:-}

2D-DFT (IDFT) of an image of size $M \times N$ can be found using FFT efficiently in $O(MN \log MN)$ time. Hence for a $N \times N$ image, 2D-DFT (IDFT) takes $O(NN \log NN) = O(N^2 \log N)$ time.
\begin{itemize}
	\item Step 1 is 2D-DFT hence takes $O(N^2 \log N)$ time.
	\item Step 2 is a pointwise multiplication and division, hence takes $O(N^2)$ time.
	\item Step 3 is 2D-IDFT, hence takes $O(N^2 \log N)$ time.
\end{itemize}
Hence the total time complexity is $O(N^2 \log N)$. \\ \\
\textbf{\Large Comparison with Pixel-wise image comparison procedure:-}

Say there are $W_x$ values of x-axis translation shift to be checked and so is $W_y$ for y-axis. For each translation shift vector, we do this:-
\begin{enumerate}
	\item Create a new image by translating $f_1$ by $(x_0, y_0)$. This takes $O(N^2)$ time.
	\item Compare each pixel of the new image with $f_2$'s pixels (or take min squares difference). This takes $O(N^2)$ time.
\end{enumerate}
Hence the total time complexity is $O(W_x W_y N^2)$. This is much larger than $O(N^2 \log N)$ for $W_x, W_y \gg \log N$.
\end{solution}

\begin{solution}{Part (b)}
	As said in Section II of the paper, $f_2(x, y) = f_1(x \cos \theta_0 + y \sin \theta_0 - x_0, \, - x \sin \theta_0 + y \cos \theta_0 - y_0)$ ($f_2$ is rotated-then-translated version of $f_1$). We have to find rotation $\theta_0$ and shift $(x_0, y_0)$ values.
	
	If $F_1(\mu, \nu)$ is the 2D-DFT of $f_1$ and so is $F_2(\mu, \nu)$ of $f_2$, then:-
	\begin{align*}
		F_2(\mu, \nu) &= \mathbf{F} [f_1(x \cos \theta_0 + y \sin \theta_0 - x_0, \, - x \sin \theta_0 + y \cos \theta_0 - y_0)] \\
		&= e^{-j2\pi (\mu x_0 + \nu y_0)/N} \mathbf{F} [f_1(x \cos \theta_0 + y \sin \theta_0, \, - x \sin \theta_0 + y \cos \theta_0)] \text{ (Fourier Shift Theorem)}\\
		&= e^{-j2\pi (\mu x_0 + \nu y_0)/N} F_1(\mu \cos \theta_0 + \nu \sin \theta_0, \, - \mu \sin \theta_0 + \nu \cos \theta_0) \text{ (Rotation property)}
	\end{align*}
	Note that the Rotation property is mentioned in the lecture slides and also in Gonzales book (pg. 241). Hence not proved here.

	Observe that magnitudes of the FTs are equal (Say $M_1, \, M_2$ are magnitudes of $F_1, \, F_2$):-
	\begin{align*}
		M_2(\mu, \nu) = |F_2(\mu, \nu)| &= |e^{-j2\pi (\mu x_0 + \nu y_0)/N} F_1(\mu \cos \theta_0 + \nu \sin \theta_0, \, - \mu \sin \theta_0 + \nu \cos \theta_0)| \\
		&= |F_1(\mu \cos \theta_0 + \nu \sin \theta_0, \, - \mu \sin \theta_0 + \nu \cos \theta_0)| \\
		&= M_1(\mu \cos \theta_0 + \nu \sin \theta_0, \, - \mu \sin \theta_0 + \nu \cos \theta_0)
	\end{align*}
	Convert to polar coordinate system: $(\mu, \nu) \rightarrow (\rho, \theta)$, $\rho = \sqrt{\mu^2 + \nu^2}$, $\theta = \tan^{-1}(\nu/\mu)$, $\mu = \rho \cos \theta, \, \nu = \rho \sin \theta$, so:-
	\begin{align*}
		\mu \cos \theta_0 + \nu \sin \theta_0 &= \rho \cos \theta \cos \theta_0 + \rho \sin \theta \sin \theta_0 = \rho (\cos \theta \cos \theta_0 + \sin \theta \sin \theta_0) = \rho \cos (\theta - \theta_0) \\
		\nu \cos \theta_0 - \mu \sin \theta_0 &= \rho \sin \theta \cos \theta_0 - \rho \cos \theta \sin \theta_0 = \rho (\sin \theta \cos \theta_0 - \cos \theta \sin \theta_0) = \rho \sin (\theta - \theta_0)
	\end{align*}
	Hence, $M_1(\mu \cos \theta_0 + \nu \sin \theta_0, \, - \mu \sin \theta_0 + \nu \cos \theta_0) = M_1(\rho, \, \theta - \theta_0)$.
	
	Then $M_2(\rho, \theta) = M_1(\rho, \theta - \theta_0)$. Hence this becomes like the problem in (a) part (but in 1D). Using the method in (a) part, we can find $\theta_0$.

	After the rotation angle $\theta_0$ is found, rotate $f_1$ by $\theta_0$ to get $f_3$. Now $f_3$ is related to $f_2$ by pure translation. Use the method in (a) part to find the translation vector $(x_0, y_0)$.

	Hence the steps are:-
	\begin{enumerate}
		\item Find 2D-DFT of $f_1$ and $f_2$.
		\item Find the magnitudes of the DFTs and convert to polar coordinates.
		\item Find the rotation angle $\theta_0$ using the method in (a) part (just in 1D).
		\item Rotate $f_1$ by $\theta_0$ to get $f_3$.
		\item Find the translation vector $(x_0, y_0)$ using the method in (a) part on $f_3$ and $f_2$.
	\end{enumerate}
\end{solution}
\end{document}
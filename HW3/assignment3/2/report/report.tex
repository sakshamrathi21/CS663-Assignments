\documentclass[12pt]{article}

\usepackage{geometry}
\geometry{a4paper, left=1in, right=1in, top=1in, bottom=1in}
\usepackage{amsmath}
\usepackage{amsmath,amsfonts,amssymb}
\usepackage{graphicx}
\usepackage{enumitem}
\usepackage{titlesec}
\usepackage{fancyhdr}
\usepackage{hyperref}
\usepackage{floatrow}
\usepackage{geometry}
\usepackage{fancyhdr}
\usepackage{empheq}
\usepackage[svgnames]{xcolor}
\usepackage{xpatch}

\makeatletter
\newcommand{\colorboxed}[1]{\fcolorbox{Black}{White}{\m@th$\displaystyle#1$}}
\xpatchcmd{\@Aboxed}{\boxed}{\colorboxed}{}{}
\makeatother

\title{{\bf CS663 Assignment 3}}
\author{Saksham Rathi, Kavya Gupta, Shravan Srinivasa Raghavan}
\date{September 2024}
\begin{document}
\maketitle
\clearpage
% \tableofcontents
% \clearpage
\section*{Question 2}
The correlation of two continuous 2D signals in the continuous domain is represented by the equation:
\begin{equation}
    h(x, y) = (f \otimes g)(x, y) = \int_{-\infty}^{\infty} \int_{-\infty}^{\infty} f(r, s)g(x+r, y+s) dr ds
\end{equation}

We need to derive the 2D fourier transform of $h(x, y)$. We know that the 2D fourier transform of a function $p(x, y)$ is given by: 

\begin{equation}
    P(u, v) = \int_{-\infty}^{\infty} \int_{-\infty}^{\infty} p(x, y) e^{-j2\pi(ux+vy)} dx dy
\end{equation}

Taking the fourier transform of $h(x, y)$, we get:

\begin{equation}
    H(u, v) = \int_{-\infty}^{\infty} \int_{-\infty}^{\infty} h(x, y) e^{-j2\pi(ux+vy)} dx dy
\end{equation}

Substituting the value of $h(x, y)$ from equation (1) into equation (3), we get:

\begin{equation}
    H(u, v) = \int_{-\infty}^{\infty} \int_{-\infty}^{\infty} \left( \int_{-\infty}^{\infty} \int_{-\infty}^{\infty} f(r, s)g(x+r, y+s) dr ds \right) e^{-j2\pi(ux+vy)} dx dy
\end{equation}

Rearranging the terms, we get:

\begin{equation}
    H(u, v) = \int_{-\infty}^{\infty} \int_{-\infty}^{\infty} f(r, s) \left( \int_{-\infty}^{\infty} \int_{-\infty}^{\infty} g(x+r, y+s) e^{-j2\pi(ux+vy)} dx dy \right) dr ds
\end{equation}


Consider $x' = x + r$ and $y' = y + v$. Thus, $x = x' - r$ and $y = y' - s$.

\begin{equation}
    H(u, v) = \int_{-\infty}^{\infty} \int_{-\infty}^{\infty} f(r, s) \left( \int_{-\infty}^{\infty} \int_{-\infty}^{\infty} g(x', y') e^{-j2\pi(u(x'-r)+v(y'-s))} dx' dy' \right) dr ds
\end{equation}

\begin{equation}
    H(u, v) = \int_{-\infty}^{\infty} \int_{-\infty}^{\infty} f(r, s) \left( \int_{-\infty}^{\infty} \int_{-\infty}^{\infty} g(x', y') e^{-j2\pi(ux'+vy')} e^{j2\pi(ur+vs)} dx' dy' \right) dr ds
\end{equation}

\begin{equation}
    H(u, v) = \int_{-\infty}^{\infty} \int_{-\infty}^{\infty} f(r, s) \left( \int_{-\infty}^{\infty} \int_{-\infty}^{\infty} g(x', y') e^{-j2\pi(ux'+vy')} dx' dy' \right) e^{j2\pi(ur+vs)} dr ds
\end{equation}

\begin{equation}
    H(u, v) = \int_{-\infty}^{\infty} \int_{-\infty}^{\infty} f(r, s) G(u,v) e^{j2\pi(ur+vs)} dr ds
\end{equation}


Let us take the complex conjugate of the equation (10):

\begin{equation}
    H^*(u, v) = G^*(u,v) \int_{-\infty}^{\infty} \int_{-\infty}^{\infty} f^*(r, s)  e^{-j2\pi(ur+vs)} dr ds
\end{equation}

\begin{equation}
    F^*(u, v) = [\int_{-\infty}^{\infty} \int_{-\infty}^{\infty} f(r, s)  e^{-j2\pi(ur+vs)} dr ds]^*
\end{equation}
\begin{equation}
    F^*(-u, -v) = \int_{-\infty}^{\infty} \int_{-\infty}^{\infty} f^*(r, s)  e^{-j2\pi(ur+vs)} dr ds
\end{equation}

Substituting the value of $F^*(-u, -v)$ from equation (12) into equation (10), we get:

\begin{equation}
    H^*(u, v) = G^*(u,v) F^*(-u, -v)
\end{equation}

\begin{equation}
    H(u, v) = G(u,v) F(-u, -v)
\end{equation}

Thus, the 2D fourier transform of the correlation of two continuous 2D signals is the product of the fourier transform of the two signals, with the other signal frequency being negated.


The correlation of two discrete 2D signals in the discrete domain is represented by the equation:

\begin{equation}
    h[m, n] = (f \otimes g)[m, n] = \sum_{r=-\infty}^{\infty} \sum_{s=-\infty}^{\infty} f[r, s]g[m+r, n+s]
\end{equation}

We need to derive the 2D DFT of $h(x, y)$. We know that the 2D DFT of a function $p(x, y)$ is given by: 

\begin{equation}
    P(u, v) = \sum_{-\infty}^{\infty} \sum_{-\infty}^{\infty} p(x, y) e^{-j2\pi(ux+vy)} 
\end{equation}

Taking the DFTof $h(x, y)$, we get:

\begin{equation}
    H(u, v) = \sum_{-\infty}^{\infty} \sum_{-\infty}^{\infty} h(x, y) e^{-j2\pi(ux+vy)}
\end{equation}

Substituting the value of $h(x, y)$ from equation (1) into equation (3), we get:

\begin{equation}
    H(u, v) = \sum_{-\infty}^{\infty} \sum_{-\infty}^{\infty} \left( \sum_{-\infty}^{\infty} \sum_{-\infty}^{\infty} f(r, s)g(x+r, y+s)\right) e^{-j2\pi(ux+vy)}
\end{equation}

Rearranging the terms, we get:

\begin{equation}
    H(u, v) = \sum_{-\infty}^{\infty} \sum_{-\infty}^{\infty} f(r, s) \left( \sum_{-\infty}^{\infty} \sum_{-\infty}^{\infty} g(x+r, y+s) e^{-j2\pi(ux+vy)}\right)
\end{equation}


Consider $x' = x + r$ and $y' = y + v$. Thus, $x = x' - r$ and $y = y' - s$.

\begin{equation}
    H(u, v) = \sum_{-\infty}^{\infty} \sum_{-\infty}^{\infty} f(r, s) \left( \sum_{-\infty}^{\infty} \sum_{-\infty}^{\infty} g(x', y') e^{-j2\pi(u(x'-r)+v(y'-s))} \right)
\end{equation}

\begin{equation}
    H(u, v) = \sum_{-\infty}^{\infty} \sum_{-\infty}^{\infty} f(r, s) \left( \sum_{-\infty}^{\infty} \sum_{-\infty}^{\infty} g(x', y') e^{-j2\pi(ux'+vy')} e^{j2\pi(ur+vs)}\right)
\end{equation}

\begin{equation}
    H(u, v) = \sum_{-\infty}^{\infty} \sum_{-\infty}^{\infty} f(r, s) \left( \sum_{-\infty}^{\infty} \sum_{-\infty}^{\infty} g(x', y') e^{-j2\pi(ux'+vy')}\right) e^{j2\pi(ur+vs)}
\end{equation}

\begin{equation}
    H(u, v) = \sum_{-\infty}^{\infty} \sum_{-\infty}^{\infty} f(r, s) G(u,v) e^{j2\pi(ur+vs)}
\end{equation}


Let us take the complex conjugate of the equation (10):

\begin{equation}
    H^*(u, v) = G^*(u,v) \sum_{-\infty}^{\infty} \sum_{-\infty}^{\infty} f^*(r, s)  e^{-j2\pi(ur+vs)}
\end{equation}

\begin{equation}
    F^*(u, v) = [\sum_{-\infty}^{\infty} \sum_{-\infty}^{\infty} f(r, s)  e^{-j2\pi(ur+vs)}]^*
\end{equation}
\begin{equation}
    F^*(-u, -v) = \sum_{-\infty}^{\infty} \sum_{-\infty}^{\infty} f^*(r, s)  e^{-j2\pi(ur+vs)}
\end{equation}

Substituting the value of $F^*(-u, -v)$ from equation (12) into equation (10), we get:

\begin{equation}
    H^*(u, v) = G^*(u,v) F^*(-u, -v)
\end{equation}

\begin{equation}
    H(u, v) = G(u,v) F(-u, -v)
\end{equation}

Thus, the 2D DFT of the correlation of two discrete 2D signals is the product of the fourier transform of the two signals, with the other signal frequency being negated.
\end{document}
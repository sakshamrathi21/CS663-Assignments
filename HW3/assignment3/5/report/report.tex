\documentclass[a4paper]{article}
\usepackage[svgnames]{xcolor}
\usepackage{amsmath,amsfonts,amssymb}
\usepackage{geometry}
\usepackage{fancyhdr}
\usepackage{graphicx}
\usepackage{titlesec}
\usepackage{caption}
\usepackage{subcaption}
\usepackage{tikz}
\usepackage{tcolorbox}
\usepackage{float} % For figure positioning
\usepackage{lipsum}
\usepackage{mdframed}
\usepackage{amsmath}
\usepackage{amsmath,amsfonts,amssymb}
\usepackage{graphicx}
\usepackage{enumitem}
\usepackage{titlesec}
\usepackage{fancyhdr}
\usepackage{hyperref}
% \usepackage{floatrow}
\usepackage{listings}
\usepackage{geometry}
\usepackage{fancyhdr}
\usepackage{empheq}
\usepackage[svgnames]{xcolor}
\usepackage{xpatch}
\usepackage{listings}

\lstdefinestyle{Matlab}{
    language=Matlab,                      % Use MATLAB language
    basicstyle=\ttfamily\footnotesize,    % Font size and style
    keywordstyle=\color{blue},            % Color for keywords
    stringstyle=\color{red},              % Color for strings
    commentstyle=\color{green!50!black},  % Color for comments
    numbers=left,                         % Line numbers on the left
    numberstyle=\tiny\color{gray},        % Style of line numbers
    stepnumber=1,                         % Line numbers step
    numbersep=5pt,                        % Distance from line numbers
    frame=single,                         % Frame around the code
    tabsize=4,                            % Set tab size
    breaklines=true,                      % Automatic line breaking
    captionpos=b                          % Position of the caption
}


\geometry{margin=0.5in}
\pagestyle{fancy}
\fancyhf{}

% Header and Footer
% \fancyhead[L]{\includegraphics[width=1.5cm]{logo.png}} % Add your logo
\fancyhead[C]{\textbf{\color{blue!70}CS663 Assignment-3}}
% \fancyhead[R]{\color{blue!70}Saksham Rathi}
\fancyfoot[C]{\thepage}

% Custom Section Color and Format
\titleformat{\section}
{\color{purple!80!black}\normalfont\Large\bfseries}
{\thesection}{1em}{}

% Beautiful Title with TikZ
\newcommand{\cooltitle}[1]{%
  \begin{tikzpicture}
    \node[fill=blue!20,rounded corners=10pt,inner sep=10pt] (box)
    {\Huge \bfseries \color{black} #1};
  \end{tikzpicture}
}

% Stylish Solution Environment with float option enabled
% Stylish Solution Environment with breakable option
% Stylish Solution Environment with mdframed
\newenvironment{solution}[2][]{%
    \begin{mdframed}[linecolor=green!60!black, linewidth=2pt, roundcorner=10pt, backgroundcolor=green!5!white, skipabove=12pt, skipbelow=12pt]%
        \textbf{\large #2} % Heading in bold and large font
        \par\noindent\rule{\textwidth}{0.4pt} % Horizontal line after heading
        \vspace{0.5em} % Small vertical space
}{%
    \end{mdframed}%
}





\title{\cooltitle{CS663 Assignment-3}}
\author{{\bf Saksham Rathi, Kavya Gupta, Shravan Srinivasa Raghavan} \\
\small Department of Computer Science, \\
Indian Institute of Technology Bombay \\}
\date{}

\begin{document}

\maketitle
\section*{Question 5}

\begin{solution}{Solution}
  The Discrete Fourier Transform (DFT) of a signal $f(x,y)$ of size $M \times N$ and the 
  Inverse Discrete Fourier Transform (IDFT) are respectively given by 
  \begin{align*}
    F(u,v) &= \frac{1}{\sqrt{MN}}\sum\limits_{x = 0}^{M - 1}\sum\limits_{y = 0}^{N - 1} f(x,y) e^{-j2\pi \left(\frac{ux}{M} + \frac{vy}{N}\right)}  \\  
    f(x,y) &= \frac{1}{\sqrt{MN}}\sum\limits_{u = 0}^{M - 1}\sum\limits_{v = 0}^{N - 1} F(u,v) e^{j2\pi \left(\frac{ux}{M} + \frac{vy}{N}\right)}
  \end{align*}
  Let $f(x,y)$ be a real function and $F(u,v)$ be it's DFT\@. By linearity of the conjugate operator and the fact that $f(t)$ 
  is real, we can show that $F^{*}(u,v) = F(-u,-v)$
  
  \begin{align*}
    F^{*}(u,v) &= {\left(\frac{1}{\sqrt{MN}}\sum\limits_{x = 0}^{M - 1}\sum\limits_{y = 0}^{N - 1} 
    f(x,y) e^{-j2\pi \left(\frac{ux}{M} + \frac{vy}{N}\right)}\right)}^{*} \\
    &= \frac{1}{\sqrt{MN}}\sum\limits_{x = 0}^{M - 1}\sum\limits_{y = 0}^{N - 1} 
    {\left(f(x,y) e^{-j2\pi \left(\frac{ux}{M} + \frac{vy}{N}\right)}\right)}^{*} \\
    &= \frac{1}{\sqrt{MN}}\sum\limits_{x = 0}^{M - 1}\sum\limits_{y = 0}^{N - 1} 
    f^{*}(x,y) {\left(e^{-j2\pi \left(\frac{ux}{M} + \frac{vy}{N}\right)}\right)}^{*} \\
    &= \frac{1}{\sqrt{MN}}\sum\limits_{x = 0}^{M - 1}\sum\limits_{y = 0}^{N - 1} 
    f(x,y) e^{j2\pi \left(\frac{ux}{M} + \frac{vy}{N}\right)} \\
    &= \frac{1}{\sqrt{MN}}\sum\limits_{x = 0}^{M - 1}\sum\limits_{y = 0}^{N - 1} 
    f(x,y) e^{j2\pi \left(\frac{-(-u)x}{M} + \frac{-(-v)y}{N}\right)} \\ 
    &= \frac{1}{\sqrt{MN}}\sum\limits_{x = 0}^{M - 1}\sum\limits_{y = 0}^{N - 1} 
    f(x,y) e^{-j2\pi \left(\frac{(-u)x}{M} + \frac{(-v)y}{N}\right)} \\
    &= F(-u,-v) \\
    \Rightarrow F^{*}(u,v) &= F(-u,-v)
  \end{align*}
  Let $f(x,y)$ be real and even. Since $f(x,y)$ is even we have $f(x,y) = f(-x,-y)$. Also, $f$ is periodic ie, $f(x + M,y + N) = f(x,y)$. Combining these 
  two equations we have,
  \begin{align}
    f(x,y) &= f(-x,-y) \nonumber \\
    f(-x,-y) &= f(M - x,N - y) \nonumber \\
    \Rightarrow \label{even}f(x,y) &= f(M - x,N - y)     
  \end{align}
  
  Also, since $f$ and $e^{-j 2\pi \left((\frac{ux}{M} + \frac{vy}{N})\right)}$ are both periodic with the same periods,
  the Discrete Fourier Transform of $f$ can also be written as 
  \begin{equation}
    \label{dft}F(u,v) = \frac{1}{\sqrt{MN}}\sum\limits_{x = 1}^{M}\sum\limits_{y = 1}^{N} f(x,y) e^{-j2\pi \left(\frac{ux}{M} + \frac{vy}{N}\right)}
  \end{equation}
  this is obtained by simply replacing any term with $x = 0$ by $x = M$ and any term with $y = 0$ by $y = N$ in the expression for $F$.

  Using equations\ref{even} and\ref{dft} we will show that $F(-u,-v) = F(u,v)$
  \begin{align}
    F(-u,-v) 
    &= \frac{1}{\sqrt{MN}}\sum\limits_{x = 0}^{M - 1}\sum\limits_{y = 0}^{N - 1} f(x,y) 
    e^{-j2\pi \left(\frac{(-u)x}{M} + \frac{(-v)y}{N}\right)}  \nonumber\\   
    &= \frac{1}{\sqrt{MN}}\sum\limits_{x = 0}^{M - 1}\sum\limits_{y = 0}^{N - 1} f(M - x,N - y) 
    e^{-j2\pi \left(\frac{(-u)x}{M} + \frac{(-v)y}{N}\right)} \text{ (using equation\ref{even})} \nonumber \\
    &= \frac{1}{\sqrt{MN}}\sum\limits_{x = 1}^{M}\sum\limits_{y = 1}^{N} f(x,y) 
    e^{-j2\pi \left(\frac{(-u)(M - x)}{M} + \frac{(-v)(N - y)}{N}\right)}  \text{ (replacing $x$ by $M - x$ and $y$ by $N - y$)}\nonumber \\
    &= \frac{1}{\sqrt{MN}}\sum\limits_{x = 1}^{M}\sum\limits_{y = 1}^{N} f(x,y) 
    e^{-j2\pi \left(\frac{(-u)(-x)}{M} + \frac{(-v)(-y)}{N}\right)}  \text{ (using periodicity of $e^{-j2\pi \left(\frac{(-u)x}{M} + \frac{(-v)y}{N}\right)}$)}\nonumber \\
    &= \frac{1}{\sqrt{MN}}\sum\limits_{x = 1}^{M}\sum\limits_{y = 1}^{N} f(x,y) 
    e^{-j2\pi \left(\frac{ux}{M} + \frac{vy}{N}\right)}  \text{ (using periodicity of $e^{-j2\pi \left(\frac{(-u)x}{M} + \frac{(-v)y}{N}\right)}$)}\nonumber \\
    &= F(u,v) \text{ (from equation\ref{dft})} \nonumber \\
    \Rightarrow F(-u,-v) &= F(u,v)
  \end{align}

  We already showed that if $f$ is real then it's DFT $F$ satisfies $F^{*}(u,v) = F(-u,-v)$. So we have $F^{*}(u,v) = F(-u,-v) = F(u,v)$
  therefore $F^{*}(u,v) = F(u,v)$ ie $F$ is real. Therefore $F$ is both real and even and we are done.
\end{solution}
\end{document}

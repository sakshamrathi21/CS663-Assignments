\documentclass[a4paper,12pt]{article}
\usepackage{amsmath,amsfonts,amssymb}
\usepackage{geometry}
\usepackage{fancyhdr}
\usepackage{graphicx}
\usepackage{titlesec}
\usepackage{xcolor}
\usepackage{caption}
\usepackage{subcaption}
\usepackage{tikz}
\usepackage{tcolorbox}
\usepackage{float} % For figure positioning
\usepackage{lipsum}
\usepackage{mdframed}
\usepackage{amsmath}
\usepackage{amsmath,amsfonts,amssymb}
\usepackage{graphicx}
\usepackage{enumitem}
\usepackage{titlesec}
\usepackage{fancyhdr}
\usepackage{hyperref}
% \usepackage{floatrow}
\usepackage{listings}
\usepackage{geometry}
\usepackage{fancyhdr}
\usepackage{empheq}
\usepackage[svgnames]{xcolor}
\usepackage{xpatch}
\usepackage{listings}

\lstdefinestyle{Matlab}{
    language=Matlab,                      % Use MATLAB language
    basicstyle=\ttfamily\footnotesize,    % Font size and style
    keywordstyle=\color{blue},            % Color for keywords
    stringstyle=\color{red},              % Color for strings
    commentstyle=\color{green!50!black},  % Color for comments
    numbers=left,                         % Line numbers on the left
    numberstyle=\tiny\color{gray},        % Style of line numbers
    stepnumber=1,                         % Line numbers step
    numbersep=5pt,                        % Distance from line numbers
    frame=single,                         % Frame around the code
    tabsize=4,                            % Set tab size
    breaklines=true,                      % Automatic line breaking
    captionpos=b                          % Position of the caption
}


\geometry{margin=0.5in}
\pagestyle{fancy}
\fancyhf{}

% Header and Footer
% \fancyhead[L]{\includegraphics[width=1.5cm]{logo.png}} % Add your logo
\fancyhead[C]{\textbf{\color{blue!70}CS663 Assignment-3}}
% \fancyhead[R]{\color{blue!70}Saksham Rathi}
\fancyfoot[C]{\thepage}

% Custom Section Color and Format
\titleformat{\section}
{\color{purple!80!black}\normalfont\Large\bfseries}
{\thesection}{1em}{}

% Beautiful Title with TikZ
\newcommand{\cooltitle}[1]{%
  \begin{tikzpicture}
    \node[fill=blue!20,rounded corners=10pt,inner sep=10pt] (box)
    {\Huge \bfseries \color{black} #1};
  \end{tikzpicture}
}

% Stylish Solution Environment with float option enabled
% Stylish Solution Environment with breakable option
% Stylish Solution Environment with mdframed
\newenvironment{solution}[2][]{%
    \begin{mdframed}[linecolor=green!60!black, linewidth=2pt, roundcorner=10pt, backgroundcolor=green!5!white, skipabove=12pt, skipbelow=12pt]%
        \textbf{\large #2} % Heading in bold and large font
        \par\noindent\rule{\textwidth}{0.4pt} % Horizontal line after heading
        \vspace{0.5em} % Small vertical space
}{%
    \end{mdframed}%
}





\title{\cooltitle{CS663 Assignment-3}}
\author{{\bf Saksham Rathi, Kavya Gupta, Shravan Srinivasa Raghavan} \\
\small Department of Computer Science, \\
Indian Institute of Technology Bombay \\}
\date{}

\begin{document}

\maketitle
\section*{Question 6}

\begin{solution}{Solution}
    Let $F(\omega) = \mathcal{F} \{f(t)\}(\omega) = \int\limits_{-\infty}^{\infty} e^{-j2\pi\omega t} f(t)dt$.
    \begin{align*}
        \mathcal{F}\{\mathcal{F}\{f(t)\}\}(\tau) 
        &= \int_{-\infty}^{\infty} e^{-j2\pi\tau\omega} 
        \left( \int_{-\infty}^{\infty} e^{-j2\pi\omega t}f(t) dt \right) d\omega \\
        &= \int_{-\infty}^{\infty} e^{-j2\pi\tau\omega} F(w) d\omega \\
        &= \int_{-\infty}^{\infty} e^{j2\pi(-\tau)\omega} F(w) d\omega
    \end{align*}

    Note that $f(\tau) = \int_{-\infty}^{\infty} e^{j2\pi\tau\omega} F(w) d\omega$ and 
    $f(-\tau) = \int_{-\infty}^{\infty} e^{j2\pi(-\tau)\omega} F(w) d\omega$. Therefore,
    \begin{align}
      \mathcal{F}\{\mathcal{F}\{f(t)\}\}(\tau) &= \int_{-\infty}^{\infty} e^{j2\pi(-\tau)\omega} F(w) d\omega = f(\tau) \nonumber \\
      \Rightarrow \mathcal{F}\{\mathcal{F}\{f(t)\}\}(\tau) &= f(-\tau) \nonumber \\
      \label{t}\Rightarrow \mathcal{F}\{\mathcal{F}\{f(t)\}\}(t) &= f(-t) 
    \end{align}
    The last step is possible because $\tau$ can be replaced by any variable and is essentially just a `formal parameter'. Let
    $\mathcal{F}\{\mathcal{F}\{f(t)\}\}(t) = \mathbb{F}(t)$
    
    From equation \ref{t} we have,
    \begin{align}
      \mathcal{F}\{\mathcal{F}\{f(t)\}\}(t) &= f(-t) \nonumber \\
      \Rightarrow \mathbb{F}(t) &= f(-t) \nonumber \\
      \Rightarrow \mathcal{F}\{\mathcal{F}\{\mathbb{F}(t)\}\}(t) &= \mathcal{F}\{\mathcal{F}\{f(-t)\}\}(t) \nonumber \\
      \Rightarrow \mathcal{F}\{\mathcal{F}\{\mathcal{F}\{\mathcal{F}\{f(t)\}\}\}\}(t) &= f(-(-t)) \text{ using eq\ref{t}}\nonumber \\
      \Rightarrow \label{ans}\mathcal{F}\{\mathcal{F}\{\mathcal{F}\{\mathcal{F}\{f(t)\}\}\}\}(t) &= f(t)
    \end{align}
    and with equation \ref{ans} we are done.
\end{solution}


\end{document}
